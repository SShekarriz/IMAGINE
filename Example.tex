% Options for packages loaded elsewhere
\PassOptionsToPackage{unicode}{hyperref}
\PassOptionsToPackage{hyphens}{url}
%
\documentclass[
  twocolumn]{article}
\usepackage{amsmath,amssymb}
\usepackage{iftex}
\ifPDFTeX
  \usepackage[T1]{fontenc}
  \usepackage[utf8]{inputenc}
  \usepackage{textcomp} % provide euro and other symbols
\else % if luatex or xetex
  \usepackage{unicode-math} % this also loads fontspec
  \defaultfontfeatures{Scale=MatchLowercase}
  \defaultfontfeatures[\rmfamily]{Ligatures=TeX,Scale=1}
\fi
\usepackage{lmodern}
\ifPDFTeX\else
  % xetex/luatex font selection
\fi
% Use upquote if available, for straight quotes in verbatim environments
\IfFileExists{upquote.sty}{\usepackage{upquote}}{}
\IfFileExists{microtype.sty}{% use microtype if available
  \usepackage[]{microtype}
  \UseMicrotypeSet[protrusion]{basicmath} % disable protrusion for tt fonts
}{}
\makeatletter
\@ifundefined{KOMAClassName}{% if non-KOMA class
  \IfFileExists{parskip.sty}{%
    \usepackage{parskip}
  }{% else
    \setlength{\parindent}{0pt}
    \setlength{\parskip}{6pt plus 2pt minus 1pt}}
}{% if KOMA class
  \KOMAoptions{parskip=half}}
\makeatother
\usepackage{xcolor}
\usepackage[margin=1in]{geometry}
\usepackage{graphicx}
\makeatletter
\def\maxwidth{\ifdim\Gin@nat@width>\linewidth\linewidth\else\Gin@nat@width\fi}
\def\maxheight{\ifdim\Gin@nat@height>\textheight\textheight\else\Gin@nat@height\fi}
\makeatother
% Scale images if necessary, so that they will not overflow the page
% margins by default, and it is still possible to overwrite the defaults
% using explicit options in \includegraphics[width, height, ...]{}
\setkeys{Gin}{width=\maxwidth,height=\maxheight,keepaspectratio}
% Set default figure placement to htbp
\makeatletter
\def\fps@figure{htbp}
\makeatother
\setlength{\emergencystretch}{3em} % prevent overfull lines
\providecommand{\tightlist}{%
  \setlength{\itemsep}{0pt}\setlength{\parskip}{0pt}}
\setcounter{secnumdepth}{-\maxdimen} % remove section numbering
%\usepackage{background}

%\backgroundsetup{
%scale=1,
%color=black,
%opacity=0.2,
%angle=0,
%pages=all,
%contents={%
%	\includegraphics[width=\paperwidth,height=\paperheight]{img/logo_transparent.png}
%	}
%}

\usepackage{lipsum}
\usepackage{graphicx}
\usepackage{fancyhdr}
\pagestyle{fancy}
\usepackage{geometry}
\usepackage{eso-pic}
\usepackage{cuted}
\usepackage[defaultfam,tabular,lining]{montserrat} %% Option 'defaultfam'
%% only if the base font of the document is to be sans serif
\usepackage[T1]{fontenc}
\renewcommand*\oldstylenums[1]{{\fontfamily{Montserrat-TOsF}\selectfont #1}}
\usepackage[fontsize=12]{scrextend}
%\setlength\voffset{-29pt}
%\geometry{hmargin=0pt}
\newcommand\BackgroundPic{%
\put(0,0){%
\parbox[b][\paperheight]{\paperwidth}{%
\vfill
\centering
\includegraphics[width=\paperwidth,height=\paperheight,%
keepaspectratio]{img/logo_for_background.png}%
\vfill
}}}
\setlength{\columnsep}{1cm}
\renewcommand{\headrulewidth}{0pt}
\renewcommand{\footrulewidth}{0pt}
\setlength\headheight{84.0pt}
\addtolength{\textheight}{-84.0pt}
\chead{\includegraphics[width=\textwidth]{img/logo_banner_cropped.png}}
\pagenumbering{gobble}
\fancyhead[R]{}

% columns.tex
\newenvironment{cols}[1][]{}{}

\newenvironment{col}[1]{\begin{minipage}{#1}\ignorespaces}{%
\end{minipage}
\ifhmode\unskip\fi
\aftergroup\usignorespacesandallpars}

\def\useignorespacesandallpars#1\ignorespaces\fi{%
#1\fi\ignorespacesandallpars}

\makeatletter
\def\ignorespacesandallpars{%
	\@ifnextchar\par
		{\expandafter\ingorespacesandallpars\@gobble}%
		{}%
}
\makeatother
\ifLuaTeX
  \usepackage{selnolig}  % disable illegal ligatures
\fi
\IfFileExists{bookmark.sty}{\usepackage{bookmark}}{\usepackage{hyperref}}
\IfFileExists{xurl.sty}{\usepackage{xurl}}{} % add URL line breaks if available
\urlstyle{same}
\hypersetup{
  hidelinks,
  pdfcreator={LaTeX via pandoc}}

\author{}
\date{\vspace{-2.5em}2023-05-26}

\begin{document}

%\AddToShipoutPicture*{\BackgroundPic}

\twocolumn[\begin{@twocolumnfalse}

\vspace{1cm}
\begin{centering}
{\fontsize{40pt}{56pt}\selectfont \textbf{Welcome to Your Microbiome!}\par}

\vspace{0.5cm}


\begin{center}\includegraphics{Example_files/figure-latex/WdCldLrgDisplay-1} \end{center}
\end{centering}
\end{@twocolumnfalse}]

\twocolumn[\begin{@twocolumnfalse}

\pagenumbering{arabic}



\section{Welcome}

Welcome to your IMAGINE microbiome report! In this report, you will find a 
description of your fecal microbiome, you will see where your sample is located
compared to other samples in the study, and at the end you'll find a more 
detailed explanation of what the microbiome is and what it does.

\vspace{1.5cm}

\end{@twocolumnfalse}]

\raisebox{-0.2\textheight}{
\begin{minipage}{0.45\textwidth}

\section{Your Sample}

\begin{small}
This bar chart shows your personal fecal bacterial community. 
Bacteria are grouped into \textbf{families} based on how closely related
they are.

The bar chart is stacked from most abundant bacterial family on the 
bottom, to least abundant on the top, with very low-abundance 
bacteria grouped together as "Other". Next to it is a chart showing
the average community of all the people who participated in the study,
and across the bottom are four anonymous individuals chosen to show
the range of variation in this study. As you can see, it's quite wide!
\end{small}

\end{minipage}
}

\newpage

\raisebox{-0.1\textheight}{
\begin{minipage}{0.45\textwidth}

\includegraphics{Example_files/figure-latex/BarPlotDisplay-1.pdf}

\end{minipage}
}

\begin{strip}
\raisebox{-0.3\textheight}{
\begin{minipage}{\textwidth}

\begin{center}\includegraphics{Example_files/figure-latex/unnamed-chunk-3-1} \end{center}

\end{minipage}
}
\end{strip}
\vspace{10cm}

\newpage

\begin{center}\includegraphics{Example_files/figure-latex/WdCldSmlDisplay-1} \end{center}

\section{You Are Here $\boldsymbol{\rightarrow}$}

\begin{small}
There were over 1,000 people participating in this study! We have 
graphed everyone together based on how similar or different their gut
microbiomes were. In the next graph, points that are closer together
are more similar to each other, and points that are farther apart are
more different. Your sample is shown in green. Remember that your 
position in this graph doesn't say anything about how healthy your
gut microbiome is. As we've seen above, there is a wide range of 
normal variation in the human gut.
\end{small}

\newpage

\raisebox{-0.2\textheight}{
\begin{minipage}{0.5\textwidth}

\section{$\boldsymbol{\leftarrow}$ More Detail}

\begin{small}
This word cloud shows your personal bacterial community at a finer 
level of detail. Within each family, bacteria can be subdivided into 
several \textbf{genera} (singular, \textbf{genus}). The size of each word
represents the relative abundance of that genus, and the colour shows
which family that genus comes from.
\end{small}

\end{minipage}
}

\raisebox{-0.4\textheight}{
\begin{minipage}{0.45\textwidth}


\begin{center}\includegraphics{Example_files/figure-latex/BrayPlotDisplay-1} \end{center}

\end{minipage}
}

\newpage

\twocolumn[\begin{@twocolumnfalse}

\section{Your Bacterial Families}

Everyone's microbiome is slightly different. In your microbiome, the four most
abundant bacterial families were \textit{Lachnospiraceae}, 
\textit{Ruminococcaceae}, \textit{Bacteroidaceae}, and 
\textit{Oscillospiraceae*}. Here is a little bit of information
about what functions these four familes perform in the gut.

\vspace{1cm}

\end{@twocolumnfalse}]

\subsection{Lachnospiraceae}

\begin{small}
This family of bacteria is found in the gut and plays an important role in the 
breakdown of complex carbohydrates. Lachnospiraceae are known for their ability 
to ferment a variety of dietary fibers, producing short-chain fatty acids 
(SCFAs) such as butyrate, propionate, and acetate. These SCFAs provide an energy
source for the body, improve gut health, and have anti-inflammatory properties. 
Research has also linked Lachnospiraceae to improved metabolic health and a 
reduced risk of obesity.
\end{small}

\subsection{Ruminococcaceae}

\begin{small}
Ruminococcaceae are involved in the fermentation of dietary fiber and the 
production of SCFAs. These bacteria also play a role in regulating the immune 
system and maintaining gut health. Research has shown that reduced levels of 
Ruminococcaceae in the gut are associated with inflammatory bowel diseases such 
as Crohn's disease and ulcerative colitis.
\end{small}

\subsection{Bacteroidaceae}

\begin{small}
Bacteroidaceae is a family of bacteria found in the gut that are involved in the
breakdown of complex carbohydrates, such as dietary fibers and resistant 
starches. These bacteria produce enzymes that can break down plant-based foods, 
making the nutrients more accessible to the body. Bacteroidaceae also play a 
role in regulating the immune system and maintaining gut health, and their 
presence has been associated with a reduced risk of inflammatory bowel diseases.
\end{small}

\subsection{Oscillospiraceae}

\begin{small}
Members of the Oscillospiraceae family of bacteria have been linked to improved 
glucose metabolism and may have a role in regulating blood sugar levels. These 
bacteria are involved in the breakdown of complex carbohydrates and the 
production of SCFAs such as butyrate, which have been shown to improve gut 
health and protect against inflammation.
\end{small}

\end{document}
